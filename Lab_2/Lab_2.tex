\documentclass{beamer}
\usetheme{CambridgeUS}

\usepackage[utf8]{inputenc}
\usepackage[russian]{babel}

\title{Компьютерная безопасность \\1 курс}
\author{Конрат Климентий Николаевич}
\institute{Балтийский федеральный университет им. Иммануила Канта}
\date{\today}

\begin{document}

\begin{frame}
\titlepage
\end{frame}

\begin{frame}
\frametitle{Описание}
\section{Задачи} 
Задачи:

\begin{itemize}
\item Узнать, что такое элептическая кривая и каковы её особенности
\item Узнать, что есть сложение точек элептической кривой
\end{itemize}

\subsection{Методы}
Методы:

\begin{itemize}
\item Обучение в университете
\item Обучение в домашних условиях
\end{itemize}

\end{frame}

\begin{frame}
\frametitle{Итог}
\section{Conclusion}
\begin{itemize}
\item Было узучено,что такое элептическая кривая и как защитные системы с ней работают
\item Были узучены правила, по которым происходит сложение точек ЭК
\item Были узучены правила построения ЭК над полем
\end{itemize}
\end{frame}

\end{document}