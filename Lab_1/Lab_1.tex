\documentclass[11pt]{article}
\usepackage{amsmath,amssymb,amsthm}
\usepackage{algorithm}
\usepackage[noend]{algpseudocode} 

%---enable russian----

\usepackage[utf8]{inputenc}
\usepackage[russian]{babel}

% PROBABILITY SYMBOLS
\newcommand*\PROB\Pr 
\DeclareMathOperator*{\EXPECT}{\mathbb{E}}

% Sets, Rngs, ets 
\newcommand{\N}{{{\mathbb N}}}
\newcommand{\Z}{{{\mathbb Z}}}
\newcommand{\R}{{{\mathbb R}}}
\newcommand{\Zp}{\ints_p} % Integers modulo p
\newcommand{\Zq}{\ints_q} % Integers modulo q
\newcommand{\Zn}{\ints_N} % Integers modulo N

% Landau 
\newcommand{\bigO}{\mathcal{O}}
\newcommand*{\OLandau}{\bigO}
\newcommand*{\WLandau}{\Omega}
\newcommand*{\xOLandau}{\widetilde{\OLandau}}
\newcommand*{\xWLandau}{\widetilde{\WLandau}}
\newcommand*{\TLandau}{\Theta}
\newcommand*{\xTLandau}{\widetilde{\TLandau}}
\newcommand{\smallo}{o} %technically, an omicron
\newcommand{\softO}{\widetilde{\bigO}}
\newcommand{\wLandau}{\omega}
\newcommand{\negl}{\mathrm{negl}} 

% Misc
\newcommand{\eps}{\varepsilon}
\newcommand{\inprod}[1]{\left\langle #1 \right\rangle}

\newcommand{\handout}[5]{
  \noindent
  \begin{center}
  \framebox{
    \vbox{
      \hbox to 5.78in { {\bf Научно-исследовательская практика} \hfill #2 }
      \vspace{4mm}
      \hbox to 5.78in { {\Large \hfill #5  \hfill} }
      \vspace{2mm}
      \hbox to 5.78in { {\em #3 \hfill #4} }
    }
  }
  \end{center}
  \vspace*{4mm}
}

\newcommand{\lecture}[4]{\handout{#1}{#2}{#3}{Автор: #4}{Простые числа и их распределение #1}}

\newtheorem{theorem}{Теорема}
\newtheorem{lemma}{Лемма}
\newtheorem{definition}{Определение}
\newtheorem{corollary}{Следствие}
\newtheorem{fact}{Факт}

% 1-inch margins
\topmargin 0pt
\advance \topmargin by -\headheight
\advance \topmargin by -\headsep
\textheight 8.9in
\oddsidemargin 0pt
\evensidemargin \oddsidemargin
\marginparwidth 0.5in
\textwidth 6.5in

\parindent 0in
\parskip 1.5ex

\begin{document}

\lecture{}{Лето 2020}{}{Конрат Климентий}

Не так давно компьютерный поиск выявил прогрессии пяти и шести последовательных простых чисел, причем члены имеют общую разницу в 30; они начинаются с простых чисел 
\begin{center}
9,843,019 \qquad и \qquad 121,174,811.
\end{center}
Мы не в состоянии обнаружить, по крайней мере в настоящее время, арифметическую прогрессию, состоящую из семи последовательных простых чисел. Когда ограничение, что простые числа должны быть последовательными, снимается, тогда можно найти бесконечно много наборов из семи простых чисел в арифметической прогрессии; одним из таких является ~7, 157, 307, 457, 607, 757, 907.

В интересах полноты картины мы могли бы упомянуть еще одну известную проблему, которая до сих пор сопротивлялась самым решительным атакам. На протяжении веков математики искали простую формулу tl+t, которая давала бы каждое простое число или, если бы это не удалось, формулу, которая не давала бы ничего, кроме простых чисел. На первый взгляд, запрос кажется достаточно скромным: найдите функцию ~f (n), область которой является, скажем, неотрицательными целыми числами и диапазон которой представляет собой некоторое бесконечное подмножество множества всех простых чисел. В Средние века широко считалось, что квадратичный многочлен 
\[
f (n) = n^2 + n + 41
\]
принимались только простые значения. Как видно из следующей таблицы, утверждение является правильным для $n$ = 0, 1, 2,\ldots, 39.

\begin{center}
\begin{tabular}{cccccc}
\hline
$n$ & $f$(n) & $n$ & $f$(n) & $n$ & $f$(n) \\
\hline
0 & 41 & 14 & 251 & 28 & 853 \\
1 & 43 & 15 & 282 & 29 & 911 \\
2 & 47 & 16 & 313 &30 & 971\\
3 & 53 & 17 & 347 & 31 & 1033\\
4 & 61 & 18 & 383 & 32 & 1097\\
5 & 71 & 19 & 421 & 33 & 1163\\
6 & 83 & 20 & 461 & 34 & 1231\\
7 & 97 & 21 & 503 & 35 & 1301\\
8 & 113 & 22 & 547 & 36 &1373\\
9 & 131 & 23 & 593 & 37 &1447\\
10 & 151 & 24 & 641 & 38 & 1523\\
11 & 173 & 25 & 691 & 39 & 1601\\
12 & 197 & 26 & 743 & &\\
13 & 223 & 27 & 797 & &\\
\hline
\end{tabular}
\end{center}

Однако эта провокационная гипотеза разбивается в случаях $n$ = 40 и $n$ = 41, где имеется коэффициент 41 :
\[
f(40) = 40 * 41 + 41 = 41^2
\]
и
\[
f (41) = 41 * 42 + 41 = 41 * 43.
\]
Следующее значение $f(42)$ = 1747 снова оказывается простым. В настоящее время неизвестно, принимает ли ~$f(n) = n^2+ n + 41$ бесконечно много простых значений для интеграла $n$.

Неспособность вышеуказанной функции быть простой производящей не является случайностью, так как легко доказать, что не существует неконстантного полинома $f(n)$ с интегральными коэффициентами, который принимает только значения для интеграла $n$. мы предполагаем, что такой полином $f(n)$ действительно существует, и спорим до тех пор, пока не будет достигнуто противоречие. Пусть 
\[
f(n)=a_kn^k+a_{k-1}n^{k-1}+\ldots+a_2n^2+a_1n+a_0
\]

где коэффициенты $a_o, a_1,\ldots, a_k$ - это все целые числа, а $a_k \ne 0$ для фиксированного значения $n$, скажем $n = n_0$, $ p=f(n_0)$ - простое число. Теперь для любого целого числа $t$ рассмотрим выражение$ f(n_o+tp)$ :

\[
f(n_0+tp)=a_k(n_0+tp)^k+\ldots+a_1(n_0+tp)+a_0=
\]
\[
=(a_kn_0^{k}+\ldots+a_1n_0+a_0)+pQ(t)= 
\]
\[
=f(n_0)+pQ(t)=
\]
\[
=p+pQ(t)=
\]
\[
p(1+Q(t))
\]

где $Q(t)$ - многочлен в $t$, имеющий интегральные коэффициенты. Наши рассуждения показывают, что $p\mid f(n_o + tp)$; следовательно, из нашего собственного предположения, что $f(n)$ принимает только простые значения, $f (n_o+tp) =p$ для любого целого числа $t$. поскольку полином степени $k$ не может принимать одно и то же значение более $k$ раз, мы получили требуемое противоречие.
В последние годы наблюдается определенный успех в поиске первичных производящих функций. Миллс доказал (1947), что существует положительное вещественное число $r$ такое, что выражение$f(n) = [r^{3^n}]$ является простым для $n$ = 1, 2, 3,\ldots (скобка указывает на наибольшую целочисленную функцию). Излишне говорить, что это строго теорема существования, и ничего не известно о действительном значении $r$.

\begin{center}
\textbf{\large{Проблемы 3.3}}
\end{center}

\textbf{1.}Убедитесь, что целые числа 1949 и 1951 являются двойными простыми числами.

\textbf{2.  } ~(a) Если 1 добавлено к произведению двойных простых чисел, докажите, что идеальный квадрат всегда получается.

\qquad(b) Покажите, что сумма двойных простых чисел $p$ и $p+ 2$ делится на 12 при условии, что $p>3$.

\textbf{3.}	Найти все пары простых чисел $p$ и $q$, удовлетворяющих $p - q = 3$.

\textbf{4.}	Сильвестр (1896) перефразировал гипотезу Гольдбаха так: «Каждое четное целое число $2n$ больше 4 представляет собой сумму двух простых чисел, одно из которых больше $n/2$, а другое меньше $3n/2$. Проверьте эту версию гипотезы для всех четных целых чисел от 6 до 76.

\textbf{5.}	В 1752 году Гольдбах представил Эйлеру следующую гипотезу: каждое нечетное целое число можно записать в виде $p + 2a^2$, где $p$ - простое число или 1 и $ a \geq 0$. Покажите, что целое число 5777 опровергает гипотезу.

\textbf{6.}	Докажите, что гипотеза Гольдбаха о том, что каждое четное целое число больше 2 является суммой двух простых чисел, эквивалентна утверждению о том, что каждое целое число больше 5 является суммой трех простых. [Подсказка: если $2n–2=
p_1+p_2$, тогда $2n = p_1+p_2+2$ и $2n+1 = p_1 + p_2 + 3$.]

\textbf{7.}	Гипотеза Лагранжа (1775) утверждает, что каждое нечетное целое число больше 5 может быть записано как сумма $p_1 + 2p_2$, где $p_1$,$p_2$ оба являются простыми числами. Подтвердите это для всех нечетных целых чисел через 75.

\textbf{8.}	Учитывая положительное целое число $n$, можно показать, что существует четное целое число $a$, которое представляется в виде суммы двух нечетных простых чисел $n$ различными способами. Убедитесь, что целые числа 60, 78 и 84 можно записать в виде суммы двух простых чисел шестью, семью и восемью способами соответственно.

\textbf{9. } ~(a) При $n> 3$ покажите, что все целые числа $n$, $n + 2$, $n + 4$ не могут быть простыми.

\qquad(б) Три целых числа $p$, $p + 2$, $p + 6$, которые являются простыми, называются \textit{простой триплет}. Найдите пять наборов простых триплетов.

\textbf{10.	}Установить, что последовательность 
\[
(n+1)!-2, (n+1)!-3, \ldots,(n+1)!-(n+1)
\]
производит $n$ последовательных составных чисел.

\textbf{11.}	Найдите наименьшее натуральное число $n$, для которого функция $f(n) = n^2 + n +17$ является составной. Сделайте то же самое для функций $g (n) = n^2+ 21n + 1$ и \\
$h(n)=3n^2+ 3n + 23.$

\textbf{12.}	Следующий результат был предположен Бертраном, но впервые доказан Чебычефом в 1850 году: для каждого натурального числа $n$> 1 существует хотя бы одно простое число $p$, удовлетворяющее $n<p<2n$. Используйте гипотезу Бертрана, чтобы показать, что $p_n<2^n$, где $p_n$ это $n$-е простое число.

\textbf{13.}	Примените тот же метод доказательства, что и в теореме 3-6, чтобы показать, что существует бесконечно много простых чисел вида 6$n$+ 5.

\textbf{14.	}Найдите простой делитель целого числа $N= 4 (3 *  7 *  11) - 1$ вида $4n + 3$. Сделайте то же самое для $N = 4(3 * 7 *  11 *  15) - 1$.

\textbf{15.}	Еще один вопрос без ответа состоит в том, существует ли бесконечное число наборов из пяти последовательных нечетных целых чисел, четыре из которых являются простыми числами. Найдите пять таких наборов целых чисел.

\textbf{16.}	Пусть последовательность простых чисел с присоединенным 1 обозначена через ~$p_0= 1$, ~$p_1 = 2$, ~$p_3= 3$ ,~$p_4= 5$ \ldots . Для каждого ~$n\geq 1$ известно, что существует подходящий выбор коэффициентов $\varepsilon_k \pm 1$, такой
\[
p_{2n}=p_{2n-1}+\sum_{k=0}^{2n-2} \varepsilon_kp_k, p_{2n+1}=2p_{2n}+\sum_{k=0}^{2n-1} \varepsilon_kp_k
\]
 
Проиллюстрировать :
13 = 1 + 2 – 3 – 5 + 7 + 11     и     17 = 1 + 2 – 3 – 5 + 7 – 11 + 2 * 13.
Определите аналогичные представления для простых чисел 23, 29, 31 и 37.

\textbf{17.}	В 1848 году де Полиньяк утверждал, что каждое нечетное целое число является суммой простого числа и степени 2. Например, 55 = 47+$2^3$ = $23+2^5$. Покажите, что целые числа 509 и 877 не соблюдают это требование.

\textbf{18.}	(а) Если $p$ простое число и $p\nmid b$, докажите, что в арифметической прогрессии
\[
a,a+b,a+2b,a+3b,\ldots
\]
 
каждый $p$-й член делится на $p$. [Подсказка: поскольку НОД($p, b$) = 1, существуют    целые числа $r$ и $s$, удовлетворяющие $pr + bs = 1$. Положим $nk= kp - as$ как для $k = 1, 2, \ldots$ и покажем, что $p| (a+b)$.]

\qquad(b) Из части (а) сделайте вывод, что если b нечетное целое число, то любой другой член в указанной последовательности является четным.

\textbf{19.}	В 1950 году было доказано, что любое целое число $n > 9$ можно записать в виде суммы различных нечетных простых чисел. Выразите целые числа 25, 69, 81 и 125 таким образом.

\textbf{20}.	Если $p$ и $p^2 + 8$ оба простые числа, докажите, что $p^3 + 4$ также простое число.

\textbf{21.	}(a) Для любого целого числа $k> 0$ установить, что арифметическая прогрессия
\[
a+b,a+2b,a+3b,\ldots,
\]
где НОД($a, b$) = 1 содержит $k$ последовательных членов, которые являются составными. [Подсказка: Положим $n = (a+b)(a+2b) \ldots (a+kb)$ и рассмотрим $k$ выражений:
\[
a+(n+1)b,a+(n+2)b,\ldots,a+(n+k)b
\]
\qquad(b) Найдите пять последовательных составных членов в арифметической прогрессии
6, 11, 16, 21, 26, 31, 36,\ldots 

\textbf{22.}	Покажите, что 13 - наибольшее простое число, которое может делить два последовательных целых числа вида $n^2 + 3$.

\textbf{23.}
(а) Среднее арифметическое простых чисел-близнецов 5 и 7 - это треугольное число 6. Существуют ли другие простые парные числа с треугольным средним?

\qquad(b) Среднее арифметическое для простых парных чисел 3 и 5 является идеальным квадратом 4. Есть ли другие простые парные числа со средним квадратом?

\textbf{24.}Определите все двойные простые числа $p$ и $q=p+2$, для которых $pq-2$ также простое число.




\end{document}