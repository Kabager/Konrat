\documentclass{beamer}
\usetheme{CambridgeUS}

\usepackage[utf8]{inputenc}
\usepackage[russian]{babel}

\title{Компьютерная безопасность \\1 курс}
\author{Конрат Климентий Николаевич}
\institute{Балтийский федеральный университет им. Иммануила Канта}
\date{\today}

\begin{document}

\begin{frame}
\titlepage
\end{frame}

\begin{frame}
\frametitle{Описание}
\section{Задачи} 
Задачи:\\
$\bullet$ Узнать, что такое элептическая кривая и каковы её особенности\\
$\bullet$ Узнать, что есть сложение точек элептической кривой\\ 
~\\
\subsection{Методы}
Методы:\\
$\bullet$ Обучение в университете\\
$\bullet$ Обучение в домашних условиях
\end{frame}

\begin{frame}
\frametitle{Итог}
\section{Conclusion}
$\bullet$ Было узучено,что такое элептическая кривая и как защитные системы с ней работают\\
$\bullet$ Были узучены правила, по которым происходит сложение точек ЭК\\
$\bullet$ Были узучены правила построения ЭК над полем\\
\end{frame}

\end{document}